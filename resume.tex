\documentclass[a4]{article}

\usepackage{hyperref}


\author{Tomas Fiers}
\def\email{tomas.fiers@gmail.com}
\def\phoneUK{+44 7735 881 358}
\def\phoneBE{+32 478 134 326}
\def\phone{\phoneUK}

\def\website{tomasfiers.net}
\def\github{tfiers}
\def\twitter{TomasFiers}


\begin{document}


{\large\phone}
\href{mailto:\email}{\email}

\makeatletter  % Make `@' a letter (normal character), so the below doesn't crash.
{\huge\@author}
\makeatother  % Make it special again (prevent leaking pkg internals).

\href{https://\website}{\large\website} \\
\href{https://github.com/\github}{github} |
\href{https://twitter.com/\twitter}{twitter}


Profile \& values:
Curiosity-driven
Not satisfied with surface-level explanations
High signal-to-noise communication
Precision \& correctness
Good user interfaces.

Motivations:
The pursuit of a proof-of-concept
The \emph{flow} during a deep session of designing / problem-solving / performance-optimization



\section{Current position}

PhD student in computational neuroscience
University of Nottingham, UK
Feb 2020  –  Dec  2023  (expected)

Topic:  Synapse-level network inference from voltage-imaging signals.
Supervisors:  \href{https://humphries-lab.org}{Mark Humphries}, Matias Ison

Topic:  Synapse-level network inference from voltage-imaging signals.
Supervisors:  [Mark Humphries][1], Matias Ison

[1]: https://humphries-lab.org

Skills trained:
Modelling \& simulation of complex systems
Scientific software design, testing, and performance-optimization (→ Library for spiking neural network simulations)
Data analysis \& experimental design
Maintaining a steady stream of scientific deliverables
Process large amounts of new literature
Communicate \& summarize complex ideas
Self-direction, proactiveness, and critical self-assesment
% Writing (at different levels)

Outreach:
\href{https://tomasfiers.net/posts/julia-for-scientists}{``Julia for Scientists''} talk
TA \& technical infrastructure for the \href{https://github.com/neural-reckoning/cosyne-tutorial-2022}{2022 Tutorial} at COSYNE (major conference for computational \& systems neuroscience) \& for the \href{https://comob-project.github.io}{COMOB project} (collaborative brain modelling)



\section{Work experience}

\href{https://www.datacamp.com}{DataCamp}
Data science training platform
Software engineering intern
Jul 2016  –  Sep 2016
Leuven, Belgium
Developed a complete data processing pipeline from scratch (Postgres, Redis, Node.js, React), that met the requirements for reliability, scalability, and performance, and that, last I heard, was still in use in production

\href{https://byteflies.com}{Byteflies}
Medical wearables startup
Data science intern
Jul 2017  –  Sep 2017
Antwerp, Belgium
Validate signal quality of motion sensors (accelerometer, gyroscope, magnetometer) against golden standard (motion capture), for use in a medical device (Python data stack).

\href{https://www.fluves.com}{Fluves}
Fiber-optic industrial monitoring
Business analysis intern
Sep 2017  –  Dec 2017
Ghent, Belgium
Preparation of business plan for joint venture with offshore wind company (Excel)



\section{Education}


BSc in engineering sciences
Computer science \& electrical engineering
Sep 2012  –  Jun 2017
KU Leuven
THE Ranking 2023: 42nd in World
Belgium

Led the teams that won the engineering design challenge for first years (2013, 1st place) and second years (2014, 2nd place; with a project on real-time speech processing)
Broad curriculum (from o.chem to semiconductor technology)


MSc in biomedical engineering
Signal processing specialization
Sep 2016  –  Jan 2019
KU Leuven
For the professional title of “burgerlijk ingenieur”  (highest-level STEM degree in the country)
Belgium

Some electives:
\href{https://github.com/tfiers/strogatz#readme}{Nonlinear dynamical systems}
Multivariate \& robust statistics
Two Bayesian modelling  courses
Three machine learning and AI courses
IT security
Computational \& Systems neuroscience

Master thesis:  Real-time signal detection for closed-loop, in-vivo neuroscience. (Detection of sharp-wave ripples on electrodes implanted in the hippocampus, \href{https://www.nerf.be/research/nerf-labs/fabian-kloosterman}{Kloosterman lab}, at \href{https://www.nerf.be}{Neuro-Electronics Research Flanders}).
Filter design \& analysis
RNN training \& evaluation
GUI development for data annotation
3D brain viewer


\section{Languages}

Expert:
Julia
Python

Productive:
OCaml
R
SQL
C
TypeScript
React.js
Kotlin
LaTeX
Command-line tools
French


\end{document}
